% Options for packages loaded elsewhere
\PassOptionsToPackage{unicode}{hyperref}
\PassOptionsToPackage{hyphens}{url}
\PassOptionsToPackage{dvipsnames,svgnames,x11names}{xcolor}
%
\documentclass[
  letterpaper,
  DIV=11,
  numbers=noendperiod]{scrartcl}

\usepackage{amsmath,amssymb}
\usepackage{iftex}
\ifPDFTeX
  \usepackage[T1]{fontenc}
  \usepackage[utf8]{inputenc}
  \usepackage{textcomp} % provide euro and other symbols
\else % if luatex or xetex
  \usepackage{unicode-math}
  \defaultfontfeatures{Scale=MatchLowercase}
  \defaultfontfeatures[\rmfamily]{Ligatures=TeX,Scale=1}
\fi
\usepackage{lmodern}
\ifPDFTeX\else  
    % xetex/luatex font selection
\fi
% Use upquote if available, for straight quotes in verbatim environments
\IfFileExists{upquote.sty}{\usepackage{upquote}}{}
\IfFileExists{microtype.sty}{% use microtype if available
  \usepackage[]{microtype}
  \UseMicrotypeSet[protrusion]{basicmath} % disable protrusion for tt fonts
}{}
\makeatletter
\@ifundefined{KOMAClassName}{% if non-KOMA class
  \IfFileExists{parskip.sty}{%
    \usepackage{parskip}
  }{% else
    \setlength{\parindent}{0pt}
    \setlength{\parskip}{6pt plus 2pt minus 1pt}}
}{% if KOMA class
  \KOMAoptions{parskip=half}}
\makeatother
\usepackage{xcolor}
\setlength{\emergencystretch}{3em} % prevent overfull lines
\setcounter{secnumdepth}{-\maxdimen} % remove section numbering
% Make \paragraph and \subparagraph free-standing
\ifx\paragraph\undefined\else
  \let\oldparagraph\paragraph
  \renewcommand{\paragraph}[1]{\oldparagraph{#1}\mbox{}}
\fi
\ifx\subparagraph\undefined\else
  \let\oldsubparagraph\subparagraph
  \renewcommand{\subparagraph}[1]{\oldsubparagraph{#1}\mbox{}}
\fi


\providecommand{\tightlist}{%
  \setlength{\itemsep}{0pt}\setlength{\parskip}{0pt}}\usepackage{longtable,booktabs,array}
\usepackage{calc} % for calculating minipage widths
% Correct order of tables after \paragraph or \subparagraph
\usepackage{etoolbox}
\makeatletter
\patchcmd\longtable{\par}{\if@noskipsec\mbox{}\fi\par}{}{}
\makeatother
% Allow footnotes in longtable head/foot
\IfFileExists{footnotehyper.sty}{\usepackage{footnotehyper}}{\usepackage{footnote}}
\makesavenoteenv{longtable}
\usepackage{graphicx}
\makeatletter
\def\maxwidth{\ifdim\Gin@nat@width>\linewidth\linewidth\else\Gin@nat@width\fi}
\def\maxheight{\ifdim\Gin@nat@height>\textheight\textheight\else\Gin@nat@height\fi}
\makeatother
% Scale images if necessary, so that they will not overflow the page
% margins by default, and it is still possible to overwrite the defaults
% using explicit options in \includegraphics[width, height, ...]{}
\setkeys{Gin}{width=\maxwidth,height=\maxheight,keepaspectratio}
% Set default figure placement to htbp
\makeatletter
\def\fps@figure{htbp}
\makeatother

\KOMAoption{captions}{tableheading}
\makeatletter
\makeatother
\makeatletter
\makeatother
\makeatletter
\@ifpackageloaded{caption}{}{\usepackage{caption}}
\AtBeginDocument{%
\ifdefined\contentsname
  \renewcommand*\contentsname{Table of contents}
\else
  \newcommand\contentsname{Table of contents}
\fi
\ifdefined\listfigurename
  \renewcommand*\listfigurename{List of Figures}
\else
  \newcommand\listfigurename{List of Figures}
\fi
\ifdefined\listtablename
  \renewcommand*\listtablename{List of Tables}
\else
  \newcommand\listtablename{List of Tables}
\fi
\ifdefined\figurename
  \renewcommand*\figurename{Figure}
\else
  \newcommand\figurename{Figure}
\fi
\ifdefined\tablename
  \renewcommand*\tablename{Table}
\else
  \newcommand\tablename{Table}
\fi
}
\@ifpackageloaded{float}{}{\usepackage{float}}
\floatstyle{ruled}
\@ifundefined{c@chapter}{\newfloat{codelisting}{h}{lop}}{\newfloat{codelisting}{h}{lop}[chapter]}
\floatname{codelisting}{Listing}
\newcommand*\listoflistings{\listof{codelisting}{List of Listings}}
\makeatother
\makeatletter
\@ifpackageloaded{caption}{}{\usepackage{caption}}
\@ifpackageloaded{subcaption}{}{\usepackage{subcaption}}
\makeatother
\makeatletter
\@ifpackageloaded{tcolorbox}{}{\usepackage[skins,breakable]{tcolorbox}}
\makeatother
\makeatletter
\@ifundefined{shadecolor}{\definecolor{shadecolor}{rgb}{.97, .97, .97}}
\makeatother
\makeatletter
\makeatother
\makeatletter
\makeatother
\ifLuaTeX
  \usepackage{selnolig}  % disable illegal ligatures
\fi
\IfFileExists{bookmark.sty}{\usepackage{bookmark}}{\usepackage{hyperref}}
\IfFileExists{xurl.sty}{\usepackage{xurl}}{} % add URL line breaks if available
\urlstyle{same} % disable monospaced font for URLs
\hypersetup{
  pdftitle={Homework 1},
  pdfauthor={STAT425, Fall 2023},
  colorlinks=true,
  linkcolor={blue},
  filecolor={Maroon},
  citecolor={Blue},
  urlcolor={Blue},
  pdfcreator={LaTeX via pandoc}}

\title{Homework 1}
\author{STAT425, Fall 2023}
\date{2023-10-05}

\usepackage{fancyhdr}

\begin{document}
\pagestyle{fancy}
\fancyhead[LH]{HW1}
\fancyhead[RH]{STAT425, Fall 2023}
% \maketitle
\ifdefined\Shaded\renewenvironment{Shaded}{\begin{tcolorbox}[enhanced, breakable, boxrule=0pt, borderline west={3pt}{0pt}{shadecolor}, interior hidden, sharp corners, frame hidden]}{\end{tcolorbox}}\fi

\begin{enumerate}
\def\labelenumi{\arabic{enumi}.}
\item
  (Serial systems) In a \emph{serial} system, components are linked
  together in such a way that the system only works if every component
  works. For example, consider a string of Christmas lights; if one
  light goes out, the whole string goes out. Suppose that one has a
  serial system with \(k\) components that all function independently of
  one another. The state of the system can be represented by a binary
  vector \(x = (x_1, \dots, x_k)\) where the coordinate \(x_i\)
  indicates whether the \(i\)th component is working. The relevant
  sample space is the set of all possible values of \(x\), that is,
  \(S = \{(x_1, \dots, x_k): x_i \in \{0, 1\}\}\), so that the system
  states are the outcomes, and the events are all possible subsets
  \(\mathcal{S} = 2^S\). Let \(E_i \in \mathcal{S}\) denote the event
  that the \(i\)th component works.

  \begin{enumerate}
  \def\labelenumii{\roman{enumii}.}
  \tightlist
  \item
    Express the sample space \(S\) as a Cartesian product.
  \item
    Express the event \(E_i\) as a set in terms of the system states
    \(x \in S\).
  \item
    List two distinct outcomes included in \(E_1\) and two distinct
    outcomes included in \(E_2\).
  \item
    Is \(\{E_i\}\) a disjoint collection? Why or why not?
  \item
    Find the number of system states \(|S|\) and the number of possible
    events \(|\mathcal{S}|\).
  \end{enumerate}

\newpage
\item
  Continuing the example in the previous problem, express each of the
  following events in terms of the collection \(\{E_i\}\).

  \begin{enumerate}
  \def\labelenumii{\roman{enumii}.}
  \tightlist
  \item
    The first component works and the second component fails.
  \item
    The first three components work.
  \item
    The system works.
  \item
    The system fails.
  \item
    Exactly one component fails.
  \end{enumerate}
\newpage
\item
  Consider the monotone sequences of sets defined by
  \(A_n = [0, 1 + \frac{1}{n})\) and \(B_n = [0, 1 - \frac{1}{n})\).

  \begin{enumerate}
  \def\labelenumii{\roman{enumii}.}
  \tightlist
  \item
    Is \(\{A_n\}\) increasing or decreasing?
  \item
    Is \(\{B_n\}\) increasing or decreasing?
  \item
    True or false:
    \(\lim_{n \rightarrow\infty} A_n = \lim_{n\rightarrow\infty} B_n\)?
    Explain. (\emph{Hint}: \(x \in \bigcup_n C_n\) just in case
    \(x \in C_n\) for at least one \(n\); similarly,
    \(x \in \bigcap_n C_n\) just in case \(x \in C_n\) for every \(n\).)
  \end{enumerate}
\newpage
\item
  Consider the ``experiment'' of rolling 2 six-sided dice, and denote
  the outcomes by pairs \((i, j)\) where
  \(i, j \in \{1, 2, 3, 4, 5, 6\}\).

  \begin{enumerate}
  \def\labelenumii{\roman{enumii}.}
  \tightlist
  \item
    Write the sample space \(S\) for this experiment, assuming the order
    of the dice does not matter (\emph{i.e.}, \((3, 2) = (2, 3)\), and
    find \(|S|\).
  \item
    If \(P(E) = 1\) whenever \(E = (1, 1)\) and \(P(E) = 0\) otherwise
    for \(E \in 2^S\), is \(P\) a valid probability measure? Why or why
    not?
  \item
    If \(P(E) = 1\) whenever \((1, 1) \in E\) and \(P(E) = 0\) otherwise
    for \(E \in 2^S\), is \(P\) a valid probability measure? Why or why
    not?
  \end{enumerate}
\newpage
\item
  (Uniform distribution) Consider the triple \((S, \mathcal{S}, P)\)
  where:

  \begin{align*}
  S &= [0, 1] \\
  \mathcal{S} &= \left\{A \subseteq S: A \text{ is a countable union or intersection of open or closed intervals or their complements}\right\} \\
  P(E) &= \int_E dx,\quad E\in\mathcal{S} \qquad\text{(i.e., total length of $E$)}
  \end{align*}
  

  \begin{enumerate}
  \def\labelenumii{\roman{enumii}.}
  \tightlist
  \item
    Show that \((S, \mathcal{S}, P)\) is a probability space by
    verifying the requisite conditions on \(\mathcal{S}\) and \(P\).
  \item
    Let \(\mathcal{C}\) denote the Cantor set. Show that
    \(P(\mathcal{C}) = 0\).
  \end{enumerate}

  \emph{Remark}: the integral \(\int_E dx\) is defined as follows:

  \begin{itemize}
  \item
    for contiguous intervals,
    \(\int_{(a, b)} dx = \int_{[a, b]} dx = \int_{[a, b)} dx = \int_{(a, b]} dx = \int_a^b dx\)
  \item
    for disjoint intervals \(E_i\),
    \(\int_{\bigcup_i E_i} dx = \sum_i \int_{E_i} dx\)
  \end{itemize}
\newpage
\item
  Let \((S, \mathcal{S}, P)\) be a probability space, and let
  \(\{E_i\}\) be a collection of events. Show that if \(\{E_i\}\) is a
  finite or countable partition of any event \(A \subseteq S\), then
  \(\sum_i P(E_i) = P(A)\).
\newpage
\item
  (Bonferroni inequality) Use results from class to show that
  \(P\left(\bigcap_{i = 1}^n E_i\right) \geq 1 - \sum_{i = 1}^n P\left(E^C\right)\).
\end{enumerate}



\end{document}
